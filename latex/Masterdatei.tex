%MAIN-DOCUMENT
%INCLUDIERT ALLER FILES DER ORDNERSTRUKTUR

% Header beinhaltet Dokumentenklasse sowie includierte Packages
%HEADER

%<---!!!!!!!!!!!!!!! MAKRO-DEFIONITIONEN; BITTE NICHT VERAENDERN !!!!!!!!!!
%<--- ARBEIT EINSEITIG
\def\makroEinseitig{
%KOMA-Script-Klasse: scrreprt
%deutsches Design, Schriftgröße 12, DIN A4
%Literaturverzeichnis und Index in Inhaltsverzerzeichnis einbinden
\documentclass[12pt,a4paper,listof=totoc,oneside]{scrreprt}
%Seitenspiegel einstellen
\usepackage[a4paper]{geometry}
\geometry{a4paper,left=30mm,right=25mm,
bottom=20mm,top=15mm,bindingoffset=2mm,
includehead,includefoot}}
% ARBEIT EINSEITIG --->

\def\makroZweiseitig{
%<--- ARBEIT ZWEISEITIG
%KOMA-Script-Klasse: scrreprt
%deutsches Design, zweiseitig
%Literaturverzeichnis und Index in Inhaltsverzerzeichnis einbinden
\documentclass[12pt,a4paper,listof=totoc,twoside, headsepline]{scrreprt}
\usepackage[a4paper]{geometry}
\geometry{a4paper,left=25mm,right=25mm,
bottom=20mm,top=15mm,bindingoffset=2mm,
includehead,includefoot}}
% ARBEIT ZWEISEITIG --->

%<--- Einstellungen Kopfzeile
\def\makroFH-Kopfzeilenstil{
\pagestyle{scrheadings} 
\setheadsepline{0.4pt}
\pagestyle{scrheadings}
\renewcommand*{\chapterpagestyle}{scrheadings}}
%Einstellungen Kopfzeile --->
%!!!!!!!!!!!!!!! MAKRO-DEFIONITIONEN; BITTE NICHT VERAENDERN !!!!!!!!!!--->


%AUSWAHL: TEXT EINSEITIG (ja/nein)
\makroEinseitig
%\makroZweiseitig

%schalte Umlaute frei
\usepackage[ngerman]{babel}
%passende Codierung
\usepackage[utf8]{inputenc}
%Seitenspiegel einzustellen
\usepackage[a4paper]{geometry}
%Mathepaket
\usepackage{amsmath}
%Symbole
\usepackage{amssymb}
%griechische Symbole
\usepackage{upgreek}
%weitere Symbole
\usepackage{pxfonts}
% Phonetischen Alphabete für LaTeX
\usepackage{tipa}
%farbige Schriften
\usepackage{color}
\usepackage{scrhack}
%Bilder fixieren
\usepackage{float}
%Grafiken einbinden
\usepackage{graphicx}
% Kopf- und Fußzeilen
\usepackage[automark,standardstyle,markusedcase]{scrpage2}
% deutsche Überschriften
\usepackage[ngerman]{translator}
% Kopfzeilenabstand festlegen
\setlength{\headheight}{10mm}
%Abb. statt Abbildung
\usepackage{caption3}
\addto\captionsngerman{
\renewcommand{\figurename}{Abb.}
\renewcommand{\tablename}{Tab.}
}
%Glossar-Pakage
\usepackage[
nonumberlist, %keine Seitenzahlen anzeigen
acronym,      %ein Abkürzungsverzeichnis erstellen
toc]          %Einträge im Inhaltsverzeichnis      
{glossaries}
\usepackage{cite}
%Glossar einschalten
\makeglossaries


%fertigen Kopfzeilenstil aktivieren
\makroFH-Kopfzeilenstil

%Zeilenabstand * 1.25 (default)
\renewcommand{\baselinestretch}{1.25}\normalsize
%(Kommentar entfernen, um Zeilenabstand
% auf 1,5-fache Groesse zu ueberschreiben)
%\renewcommand{\baselinestretch}{1.50}\normalsize

%Startpunkt jedes LaTex-Dokuments
\begin{document}

%<--- Benutze Glossar
\input{Glossar/Glossar}
\glsaddall 
%Benutze Glossar --->

%Verwende Muster-Titelseite 
%Titelseite
% Seitennummer aus
\thispagestyle{empty}
\begin{titlepage}
	\vspace{3cm}

%%\begin{center}
%%	\Huge	
%%	HOCHSCHULE LANDSHUT \\
%%	\Large
%%	FAKULTÄT INFORMATIK
%%\end{center}

%%\vspace{1cm}

\begin{center}
	\includegraphics[scale=0.8]{Titelseite/hl-logo.pdf}  
\end{center}

\vspace{2.5cm}

\begin{center}
  \Large FAKULTÄT INFORMATIK
\end{center}

\vspace{1cm}
\begin{center}
	\Huge
	\textbf{Bachelorarbeit/Masterarbeit}\\
\end{center}

\vspace{1cm}

\begin{center}
	\Large
	\textsc{Dies ist der Titel meiner Bacherlor/Masterarbeit}\\
\end{center}

\vspace{1.5cm}

\begin{center}
	\Large
	Vorname Nachname
\end{center}

\vspace{4cm}
\begin{center}
	\large
	Betreuer: Prof. Max Mustermann
\end{center}

\end{titlepage}

%Füge leere Seite ein (optional)
\input{leereSeite}
%Verwende Muster-Erklärung zur selbstständigen Arbeit 
\include{Erklaerung/Erklaerung}
%Füge leere Seite ein (optional)
\input{leereSeite}
%Verwende Muster-Abstract 
\begin{abstract}
\begin{center}
\Huge
\emph{\textbf{Zusammenfassung/Abstract}}
\end{center}
\normalsize
\vspace{15mm}
\textit{Eine Inhaltsangabe oder Zusammenfassung ist eine Übersicht über den wesentlichen Inhalt eines Textes, Filmes oder Ereignisses. Gebräuchliche Formen von Inhaltsangaben sind das Inhaltsverzeichnis, das Abstract und andere Formen dokumentarischer Referate. Auch die englische Bezeichnung Summary ist in wissenschaftlichen Arbeiten üblich. Die Inhaltsangabe ist in der DIN-Norm DIN 1426 genormt.}
\end{abstract}


%Liste Menüpunkte als Inhaltsverzeichnis
\tableofcontents
\setcounter{page}{1}

%Jedes Kapitel in eigener Ordnerstruktur 
%1.Kapitel ist die Zusammenfassung
\input{Kapitel2/Kapitel2-Text/Kapitel2-Text}
\input{Kapitel3/Kapitel3-Text/Kapitel3-Text}
\input{Kapitel4/Kapitel4-Text/Kapitel4-Text}
\input{Kapitel5/Kapitel5-Text}

% <--- BibTex mit Stil alpha
\bibliography{Bibliothek}{}
\bibliographystyle{alpha}
% BibTex mit Stil alpha --->
 
% Abbildungsverzeichnis, Tabellenverzeichnis und Glossar ausgeben
\listoffigures
\listoftables
\printglossaries 

%Ende jedes LaTex-Dokuments
\end{document}

