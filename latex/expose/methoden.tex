\chapter*{Methoden}
Zur Erfüllung der obengenannten Ziele ergeben sich folgende Aufgaben:
\section*{Recherche aktueller unpaired Image-To-Image Translation Methoden}
CycleGANs wurden erstmals 2017 vorgestellt, seitdem sind zahlreiche weitere unpaired Image-To-Image Translation Methoden erschienen.
Es muss eine Suche und Auswahl einer (bzw. mehrerer) dieser Methoden durchgeführt werden, die im Rahmen dieser Arbeit implementiert werden.
Hierzu können moderne Suchmaschinen wie semanticscholar oder google scholar verwendet werden.
\section*{Datenbeschaffung}
Sowohl für die Image-To-Image Translation als auch für die Klassifizierung werden Trainingsdaten, Bilder von Stromzählern, benötigt.
Einerseits bietet der UFPR-AMR Datensatz 2000 Bilder. Sollte sich diese Datenmenge als zu gering herausstellen, wäre auch eine Umfrage denkbar,
bei der Studenten der Hochschule Landshut per Rundmail um ein mit dem Smartphone geschossenes Foto ihres Stromzählers gebeten werden.
\section*{Erstellung eines Generators synthetischer Zählerstandbilder}
Der erste Teil der obenbeschriebenen Pipeline ist ein Bildgenerator, der aus einer beliebigen Ziffernfolge ein synthetisches Bild eines Stromzählers generiert.
Diese Bilder haben nicht den Anspruch realistisch sein, und können also durch Skripts generiert werden, etwa mit openCV oder als SVG-Bild.
\section*{Implementierung einer unpaired Image-To-Image Translation Methode}
Die gewählte Image-To-Image Translation Methode soll implementiert und jeweils mit generierten synthetischen und realen Bildern aus den Trainigsdaten trainiert werden,
sodass das Modell synthetische Bilder in realistische überführt.
\section*{Evaluation}
Mit der fertigen Pipeline können aus beliebigen Ziffernfolgen realistische Bilder von Stromzählern generiert und damit der bestehende Datensatz erweitert werden.
Die Qualität hiervon lässt sich dann bestimmen, indem man zwei Instanzen desselben Klassifizierermodells vergleicht: Eines trainiert man nur mit den ursprünglich
vorhandenen Daten, das andere zusätzlich mit den generierten.
Idealerweise weist das Modell, welches zusätzlich mit generierten Daten trainiert wurde, eine höhere Erkennungsrate auf.
Im ersten Anlauf wird aber eher erwartet, dass das Ergebnis nicht so positiv ausfällt. Deshalb seien obige Aufgaben nicht als einmalige Arbeitsschritte anzusehen, 
sie sollen eher in einem iterativem Prozess abgearbeitet werden. Ist das Endergebnis nicht zufriedenstellend, sollen sowohl die Auswahl der verwendeten
Image-To-Image Translation Methode, als auch die Trainigsdaten, der Generator synthetischer Bilder und die Implementierung der Image-To-Image Methode überdacht werden.