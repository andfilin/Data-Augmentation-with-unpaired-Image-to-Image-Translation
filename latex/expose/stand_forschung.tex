\chapter*{Stand der Forschung}
Mit GANs\cite{gan} können beginnend mit einem Anfangsdatensatz weitere (realistische) Bilder erzeugt werden.
Dies wird durch die Verwendung von zwei Neuronalen Netzen erreicht, die jeweils gegensätzliche Ziele verfolgen:
Der Generator versucht möglichst realistische Bilder zu erzeugen, während der Diskriminator versucht reale Bilder von generierten zu unterscheiden.
Auf diese Art erzeugte Bilder wurden bereits dazu verwendet, zusätzliche Trainingsdaten für Neuronale Netze zu erzeugen.
Yi et al\cite{medic} sammelten in Ihrer Studie Anwendungen von GANs im medizinischen Bereich. GANs sind hier besonders attraktiv, da Trainingsdaten oft
schwer zu beschaffen sind, etwa weil die Privatsphäre von Patienten betroffen ist.
So steigerten beispielsweise Frid-Adar et al\cite{liver} die Erkennungsleistung verschiedener Leberläsionen,
indem sie zu jeder der drei Klassen synthetische Bilder mit GANs erzeugten.

CycleGAN\cite{cyclegan} ermöglicht eine Abbildung von einer Bildmenge (Domäne) in eine andere, etwa ein Foto einer Landschaft im Sommer in ein Foto im Winter.
Image-to-Image Translation gab es zwar schon vor CycleGAN, bisher benötigte man aber eine Menge korrespondierende Bildpaare beider Domänen, zum Beispiel jeweils die selbe Landschaft im Sommer und im Winter (paired Image-to-Image Translation).
Bei CycleGAN benötigt man keine korrespondierenden Bildpaare mehr, es reichen zwei Mengen unabhängiger Bilder beider Domänen (unpaired Image-to-Image Translation).

Wie oben erwähnt verwendeten Wang et al\cite{lplate} CycleGAN, um synthetische Bilder von Nummernschildern in realistische zu überführen. Damit generierten sie 
eine Menge realistischer Trainingsdaten, deren Inhalt (Ziffernfolge) vorgegeben werden konnte.
Diese Herangehensweise wird im Rahmen dieser Masterarbeit auf das Problem der Erkennung von Stromzählerwerten angewendet.
