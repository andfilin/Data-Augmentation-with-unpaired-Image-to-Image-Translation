\chapter*{Zielsetzung}
Als Ergebnis der Arbeit soll eine Pipeline entstehen,
die als Input eine beliebige Ziffernfolge nimmt und daraus ein realistisches Bild eines Stromzählers generiert.
Das Ziel ist es mithilfe von unpaired Image-to-Image Translation Bilder zu generieren, die, wenn als zusätzliche Trainigsdaten verwendet, die Erkennungsleistung eines Modells
 zum Ablesen von Stromzählerwerten steigern.
Die Erfüllung dieses Ziels kann anhand eines Testdatensatzes wie UFPR-AMR\cite{ufpr}, einer Sammlung von Stromzählerbildern, gemessen werden.
Hierzu vergleicht man die Erkennungsleistung zweier sonst identischer Modelle, von denen eines zusätzlich mit den generierten Daten trainiert wurde.

Der erhoffte Nutzen der Arbeit, sollte dieses Ziel erreicht werden können, ist es die Realisierung eines Gerätes zum Ablesen 
von Stromzählerwerten zu unterstützen. Das Gesetz verlangt, dass bis 2032 in jedem Haushalt ein Stromzähler verbaut ist, der
in der Lage ist automatisch Zählerwerte abzulesen und über das Internet zu versenden.\cite{bsi}
Die Kosten für die Anschaffung eines solchen Smartmeters würde der Verbraucher selbst tragen.
Eine Alternative zum Austauchen alter Zähler könnte ein Gerät sein, welches mit einer fest montierten Kamera ein Bild des
Zählers macht, daraus den Zählerstand mithilfe eines Neuronalen Netzes auf einem Prozessor ermittelt und schließlich über 
eine Kommunikationseinheit versendet.

Konkret ergeben sich zur Realisierung der Pipeline folgende Aufgaben:
\begin{itemize}
  \item Recherche und Auswahl aktueller unpaired Image-to-Image Methoden
  \item Implementierung eines Bildgenerators, der aus Ziffernfolgen synthetische Zählerstandbilder erzeugt
  \item Implementierung einer (oder mehrerer) unpaired Image-to-Image Methode, die synthetische Zählerstandbilder in realistische Bilder überführt
  \item Evaluierung mithilfe eines Benchmarks
\end{itemize}
