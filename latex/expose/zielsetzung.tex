\chapter*{Zielsetzung}
Als Ergebnis der Arbeit soll eine Pipeline entstehen,
die als Input eine beliebige Zahlenfolge nimmt und daraus ein realistisches Bild eines Stromzählers generiert.
Das Ziel ist es mithilfe von Image-to-Image Translation Bilder zu generieren, die die Erkennungsleistung eines Modells
 zum Ablesen von Stromzählerwerten steigert.
Die Erfüllung dieses Ziels kann anhand eines Benchmarks wie UFPR-AMR\cite{ufpr}, einer Sammlung von Stromzählerbildern, gemessen werden.
Wenn das Ergebnis positiv ist und die generierten Trainingsdaten eine Steigerung der Erkennungsleistung zur Folge hat,
könnte das Verfahren dabei helfen, einen realen Zählerstandleser zu entwickeln\footnote{todo: noch etwas über realen Nutze, gesetzliche Pflicht, Strom und kosten sparen...}

Konkret ergeben sich folgende Aufgaben:
\begin{itemize}
  \item Recherche und Auswahl aktueller unpaired Image-to-Image Methoden
  \item Implementierung eines Bildgenerators, der aus Ziffernfolgen synthetische Zählerstandbilder erzeugt
  \item Implementierung einer unpaired Image-to-Image Methode, die synthetische Zählerstandbilder in realistische Bilder überführt
  \item Evaluierung mithilfe eines Benchmarks
\end{itemize}

(sollen selbst Daten gesammelt werden, oder reicht der UFPR-AMR Datensatz?)
(soll ein Netzwerk zur Erkennung von Stromzählerwerten selbst erstellt werden, oder gibt es ein vorhandenes?)